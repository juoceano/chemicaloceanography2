\RequirePackage{fix-cm}  % Fix Font shape `OT1/cmr/m/n' size substitution.
%-----------------------------------------------------------------------------%
%	Packages & Other Configurations
%-----------------------------------------------------------------------------%
\documentclass[a4paper,10pt]{article}
\usepackage[top=0.5in, bottom=0.5in, left=0.9in, right=0.3in]{geometry}

\usepackage[utf8]{inputenc} %add acents
\usepackage{setspace} % command \doublespacing etc...
\usepackage{lineno} % number lines
\usepackage{epsf,epsfig} % includegraphics [pdf, png etc]
\usepackage{amsmath} %adicionei esse pacote pra vc poder usar o draft%
\usepackage{textcomp} %símbolos de texto
\usepackage{natbib} % bibtex - adicionar referencia
% \usepackage{url} % for bibtex - configuracoes de urls
\usepackage{tabularx} % for tables
\usepackage[hidelinks]{hyperref}  % Add URL links.
% \usepackage[bookmarks=false,colorlinks=true,urlcolor={green},linkcolor={green},pdfstartview={XYZ null null 1.22}]{hyperref} %all references
\usepackage{multicol}


%-----------------------------------------------------------------------------%
%	Adicionar a Watermark
%-----------------------------------------------------------------------------%
\usepackage{draftwatermark}
\SetWatermarkAngle{45}
\SetWatermarkLightness{0.9}
\SetWatermarkFontSize{5cm}
\SetWatermarkScale{0.5}
\SetWatermarkText{Questionário Inicial}

%-----------------------------------------------------------------------------%
%	Informações sobre o PDF
%-----------------------------------------------------------------------------%

\pdfinfo{%
  /Title    (QUI155 - Questionário Inicial)
  /Author   (Ju Leonel)
  /Creator  (Ju Leonel)
  /Producer (Ju Leonel)
  /Subject  (Questionário Inicial)
  /Keywords (Questionário Inicial, oceanografia química)}

%-----------------------------------------------------------------------------%
%	Documento
%-----------------------------------------------------------------------------%
\title{QUI155 Oceanografia Química II - Questionário Inicial}
\author{\vspace{-10ex}}
\date{\vspace{-10ex}}

\begin{document}

  \maketitle
  %\doublespacing
  \onehalfspace

%   \begin{tabular*} {0.9\textwidth}{@{\extracolsep{\fill} } l l}
%     \hline
%     Professora: Juliana Leonel & Atendimento: Sextas-feiras \\
%     E-mail: \href{mailto:jleonel@ufba.br}{jleonel@ufba.br} & Horário atendimento: 13:00-14:00 \\
%     Aulas: Quintas e sextas-feiras & Local atendimento: IGEO - Sala 10 - 2\textsuperscript{o} andar\\
%     Horário- Aulas: 7:00 - 10:40 & Homepage: \url{http://juoceano.github.io/geochemistry}\\
%     \hline
%   \end{tabular*}
%
%   \vspace{3ex}

Para cada um dos termos abaixo classifique seu nível de conhecimento/familiarização de acordo com a legenda abaixo:

\begin{multicols}{2}

I. Nunca ouvi falar.

II. Ouvi falar, mas não lembro o que é.

III. Tenho uma ideia do que é.

IV. Sei o que é e posso explicar o que é.

V. Sei o que é, posso explicar e consigo usar isso no meu trabalho.

\end{multicols}

\begin{multicols}{2}

1. análise quantitativa $\rule{1cm}{0.15mm}$

2. análise qualitativa $\rule{1cm}{0.15mm}$

3. controle de qualidade $\rule{1cm}{0.15mm}$

4. branco de campo $\rule{1cm}{0.15mm}$

5. branco do equipamento $\rule{1cm}{0.15mm}$

6. branco do método de análise $\rule{1cm}{0.15mm}$

7. material de referência $\rule{1cm}{0.15mm}$

8. exatidão $\rule{1cm}{0.15mm}$

9. precisão $\rule{1cm}{0.15mm}$

10. erro sistemático $\rule{1cm}{0.15mm}$

11. erro aleatório $\rule{1cm}{0.15mm}$

12. repetibilidade $\rule{1cm}{0.15mm}$

13. reprodutibilidade $\rule{1cm}{0.15mm}$

14. algorismos significativos $\rule{1cm}{0.15mm}$

15. média $\rule{1cm}{0.15mm}$

16. mediana $\rule{1cm}{0.15mm}$

17. desvio padrão $\rule{1cm}{0.15mm}$

18. variância $\rule{1cm}{0.15mm}$

19. população $\rule{1cm}{0.15mm}$

20. amostra $\rule{1cm}{0.15mm}$

21. distribuição normal $\rule{1cm}{0.15mm}$

22. intervalo de confiança $\rule{1cm}{0.15mm}$

23. t student $\rule{1cm}{0.15mm}$

24. teste F $\rule{1cm}{0.15mm}$

25. gravimetria $\rule{1cm}{0.15mm}$

26. titrimetria $\rule{1cm}{0.15mm}$

27. espectrofotometria $\rule{1cm}{0.15mm}$

28. cromatografia $\rule{1cm}{0.15mm}$

29. condutividade $\rule{1cm}{0.15mm}$

30. fluorescência $\rule{1cm}{0.15mm}$

31. curva de quantificação $\rule{1cm}{0.15mm}$

32. coeficiente de variação $\rule{1cm}{0.15mm}$

33. teste Q $\rule{1cm}{0.15mm}$

34. teste de Grubbs $\rule{1cm}{0.15mm}$

35. sensibilidade $\rule{1cm}{0.15mm}$

36. seletividade $\rule{1cm}{0.15mm}$

37. limite de detecção $\rule{1cm}{0.15mm}$

38. limite de quantificação $\rule{1cm}{0.15mm}$

39. Trombeta de Horwitz $\rule{1cm}{0.15mm}$

40. métodos eletroanalíticos $\rule{1cm}{0.15mm}$

41. métodos espectroanalíticos $\rule{1cm}{0.15mm}$

42. espectroscopia atômica $\rule{1cm}{0.15mm}$

43. espectroscopia molecular $\rule{1cm}{0.15mm}$

44. espectroscopia de ressonância magnética $\rule{1cm}{0.15mm}$

45. espectroscopia de massas $\rule{1cm}{0.15mm}$

\end{multicols}

Para cada um dos seguintes conceitos/ferramentas use a legenda abaixo para descrever sua experiência:

I.Nunca ouvi falar

II. Vi em outra disciplina. Qual?

III. Li sobre o assunto.

IV. Usei em um trabalho (disciplina, estágio, etc)

1. média $\rule{1cm}{0.15mm}$

2. mediana $\rule{1cm}{0.15mm}$

3. variância $\rule{1cm}{0.15mm}$

4. t student $\rule{1cm}{0.15mm}$

5. gravimetria $\rule{1cm}{0.15mm}$

6. análise titrimétrica $\rule{1cm}{0.15mm}$

7. espectrofotometria $\rule{1cm}{0.15mm}$

8. teste de Grubbs $\rule{1cm}{0.15mm}$

9. exercício de intercalibração laboratorial $\rule{1cm}{0.15mm}$

10. desvio padrão $\rule{1cm}{0.15mm}$

11. coeficiente de variação $\rule{1cm}{0.15mm}$


%\clearpage %termina o texto e tudo que estiver flutuante se encaixa ali
%\newpage % começa uma nova página
%\bibliographystyle{chicago} % estilo que vai sair a bibliografia (posso usar outros)
%\bibliography{references_ju} %entre chaves vai o nome do arquivo de referencias

\end{document}
