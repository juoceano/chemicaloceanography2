\RequirePackage{fix-cm}  % Fix Font shape `OT1/cmr/m/n' size substitution.
%-----------------------------------------------------------------------------%
%	Packages & Other Configurations
%-----------------------------------------------------------------------------%
\documentclass[a4paper,10pt]{article}
\usepackage[top=0.5in, bottom=0.5in, left=0.9in, right=0.3in]{geometry}

\usepackage[utf8]{inputenc} %add acents
\usepackage{setspace} % command \doublespacing etc...
\usepackage{lineno} % number lines
\usepackage{epsf,epsfig} % includegraphics [pdf, png etc]
\usepackage{amsmath} %adicionei esse pacote pra vc poder usar o draft%
\usepackage{textcomp} %símbolos de texto
\usepackage{natbib} % bibtex - adicionar referencia
% \usepackage{url} % for bibtex - configuracoes de urls
\usepackage{tabularx} % for tables
\usepackage[hidelinks]{hyperref}  % Add URL links.
% \usepackage[bookmarks=false,colorlinks=true,urlcolor={green},linkcolor={green},pdfstartview={XYZ null null 1.22}]{hyperref} %all references
\usepackage{multicol}


%-----------------------------------------------------------------------------%
%	Adicionar a Watermark
%-----------------------------------------------------------------------------%
\usepackage{draftwatermark}
\SetWatermarkAngle{45}
\SetWatermarkLightness{0.9}
\SetWatermarkFontSize{5cm}
\SetWatermarkScale{0.5}
\SetWatermarkText{Exercício 4}

%-----------------------------------------------------------------------------%
%	Informações sobre o PDF
%-----------------------------------------------------------------------------%

\pdfinfo{%
  /Title    (QUI155 - Exercício 4)
  /Author   (Ju Leonel)
  /Creator  (Ju Leonel)
  /Producer (Ju Leonel)
  /Subject  (Questionário Inicial)
  /Keywords (Exercício 4, oceanografia química)}

%-----------------------------------------------------------------------------%
%	Documento
%-----------------------------------------------------------------------------%
\title{QUI155 Oceanografia Química II - Exercício 4}
\author{\vspace{-10ex}}
\date{\vspace{-10ex}}

\begin{document}

  \maketitle
  %\doublespacing
  \onehalfspace

  \begin{itemize}
  
    \item[1.] Hipoteticamente, qual seria a melhor maneira de processar uma amostra antes da análise? Na prática, isso é possível? Justifique sua resposta.
    
    \item[2.] Nas etapas de processamento das amostras, explique cada uma das etapas abaixo e destaque a sua importância:
      \item[(a)] Moagem;
      \item[(b)] Filtração;
      \item[(c)] Secagem;
    
    \item[3.] A preservação das amostras desde a coleta até o início do processamento é uma etapa muito importante para evitar perda dos analitos de interesse. Cite e explique 3 formas de preservar a amostra.
    
    \item[4.] O processamento de amostras possui, entre outras, etapas de digestão/extração e purificação. Explique, em linhas gerais, o que são essas etapas e por que elas devem ser realizadas.
    
    \item[5.] Quais as limitações quando a matriz de trabalho é água do mar? 
    
    \item[6.] Quando amostras estão sendo processadas, tanto para compostos orgânicos como inorgânicos, algumas cuidados devem ser tomadas para prevenir a  contaminação. Quais são esses cuidados?
  
  \end{itemize}


%\clearpage %termina o texto e tudo que estiver flutuante se encaixa ali
%\newpage % começa uma nova página
%\bibliographystyle{chicago} % estilo que vai sair a bibliografia (posso usar outros)
%\bibliography{references_ju} %entre chaves vai o nome do arquivo de referencias

\end{document}
